idea about teaching/motivating continued fractions

trying to estimate $\pi$ by measurement. can do better than wrapping a tape
measure around a cylindrical object and estimating length. use an unmarked
string. carefully comparing circumference to diameter will show that $\pi$ is a bit less than 3 plus 1/7. This is already a more accurate estimate than we are
likely to get with the tape measure approach. But we can do better; do the same trick with the difference between 1/7 and $\pi - 3$ to get the next convergent
\[
\frac{1}{7 + \frac{1}{16}}
\]
which is a very accurate estimate.

Other ways to estimate $\pi$. Buffon's needle is classic, but our interest is in deterministic methods. Take a quarter circle inside a square and count how many grid points
are inside/outside circle. But this is not a physical measurement method, it is purely
mathematical. Physical analogue is to cut a piece of wood and weigh the pieces; then can try the same approach that we used with string lengths. Actual pieces of solid material are not so good for figuring out that one is about seven times as heavy as the other -- could use balance scale to obtain equal weight of liquid which is easier to manipulate. Or can use first piece as a template to make seven copies. In any case need a reasonably accurate balance scale (how much do such things cost?). Well, for comparing liquids do not need a scale, can convert volume to distance by having two identical vessels.

Any way to cut off a length of string equal to $e$? Or can this only be approximated by a limiting process?

Something similar with volumes of liquid instead of strings? Areas even?

Mechanical device for rectifying a curve?

``Hands on Math'' is catchy title for youTube channel ...? These would make good videos, but producing decent video might be cumbersome.
